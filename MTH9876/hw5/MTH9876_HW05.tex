%%%%%%%%%%%%%%%%%%%%%%%%%%%%%%%%%%%%%%%%%
% Short Sectioned Assignment
% LaTeX Template
% Version 1.0 (5/5/12)
%
% This template has been downloaded from:
% http://www.LaTeXTemplates.com
%
% Original author:
% Frits Wenneker (http://www.howtotex.com)
%
% License:
% CC BY-NC-SA 3.0 (http://creativecommons.org/licenses/by-nc-sa/3.0/)
%
%%%%%%%%%%%%%%%%%%%%%%%%%%%%%%%%%%%%%%%%%

%----------------------------------------------------------------------------------------
%	PACKAGES AND OTHER DOCUMENT CONFIGURATIONS
%----------------------------------------------------------------------------------------

\documentclass[paper=a4, fontsize=11pt]{scrartcl} % A4 paper and 11pt font size

\usepackage[T1]{fontenc} % Use 8-bit encoding that has 256 glyphs
\usepackage{fourier} % Use the Adobe Utopia font for the document - comment this line to return to the LaTeX default
\usepackage[english]{babel} % English language/hyphenation
\usepackage{amsmath,amsfonts,amsthm, amssymb} % Math packages
\usepackage{enumitem}
\usepackage{cancel}
\usepackage{graphicx} % Required to insert images
\usepackage{color}
\definecolor{dkgreen}{rgb}{0,0.6,0}
\definecolor{gray}{rgb}{0.5,0.5,0.5}
\definecolor{mauve}{rgb}{0.58,0,0.82}
\usepackage{listings}
\lstset{frame=tb,
  language=Python,
  aboveskip=3mm,
  belowskip=3mm,
  showstringspaces=false,
  columns=flexible,
  basicstyle={\small\ttfamily},
  numbers=none,
  numberstyle=\tiny\color{gray},
  keywordstyle=\color{blue},
  commentstyle=\color{dkgreen},
  stringstyle=\color{mauve},
  breaklines=true,
  breakatwhitespace=true
  tabsize=3
}
\usepackage{pgfplotstable}% For inserting csv table
\usepackage{MnSymbol,wasysym}
\usepackage{sectsty} % Allows customizing section commands
\allsectionsfont{\centering \normalfont\scshape} % Make all sections centered, the default font and small caps

\usepackage{fancyhdr} % Custom headers and footers
\pagestyle{fancyplain} % Makes all pages in the document conform to the custom headers and footers
\fancyhead{} % No page header - if you want one, create it in the same way as the footers below
\fancyfoot[L]{} % Empty left footer
\fancyfoot[C]{} % Empty center footer
\fancyfoot[R]{\thepage} % Page numbering for right footer
\renewcommand{\headrulewidth}{0pt} % Remove header underlines
\renewcommand{\footrulewidth}{0pt} % Remove footer underlines
\setlength{\headheight}{13.6pt} % Customize the height of the header

\numberwithin{equation}{section} % Number equations within sections (i.e. 1.1, 1.2, 2.1, 2.2 instead of 1, 2, 3, 4)
\numberwithin{figure}{section} % Number figures within sections (i.e. 1.1, 1.2, 2.1, 2.2 instead of 1, 2, 3, 4)
\numberwithin{table}{section} % Number tables within sections (i.e. 1.1, 1.2, 2.1, 2.2 instead of 1, 2, 3, 4)

\setlength\parindent{0pt} % Removes all indentation from paragraphs - comment this line for an assignment with lots of text

\newcommand{\vast}{\bBigg@{4}}
\newcommand{\Vast}{\bBigg@{5}}

%----------------------------------------------------------------------------------------
%	TITLE SECTION
%----------------------------------------------------------------------------------------

\newcommand{\horrule}[1]{\rule{\linewidth}{#1}} % Create horizontal rule command with 1 argument of height

\title{	
\normalfont \normalsize 
\textsc{Baruch, MFE} \\ [25pt] % Your university, school and/or department name(s)
\horrule{0.5pt} \\[0.4cm] % Thin top horizontal rule
\huge MTH 9876 Assignment Five\\  % The assignment title
\horrule{2pt} \\[0.5cm] % Thick bottom horizontal rule
}


\author{Zhou, ShengQuan} % Your name

\date{\normalsize\today} % Today's date or a custom date

\begin{document}
	


\maketitle % Print the title

\newpage

\section{A VaR Model for a Portfolio of Equities}
In this problem you will develop a VaR model for a portfolio of equities, based on histrical simulations 
(this is essentially the industry standard methodology). It is somewhat data analysis intensive, 
so give yourself enough time to work on this problem.\\

\textbf{(i)} Download the historical daily prices since January 1, 2004 of the following names: 
AAPL, HAL, INTC, JPM, PFE, CAT, DE, MSFT, WMT, C. You can use, e.g., Yahoo Finance or Bloomberg as the source
of data. Make sure to use the \textit{adjusted} prices, this way you don't have to worry about dividends and/or splits. 
Cosider a model portfolio that consists of 10,000 shares of each of these shares.\\
\textit{Solution}: See attached spreadsheet \textit{Problem1 VaR Model Equity Portfolio.xlsx} data tab.\\

\textbf{(ii)} Assuming the MPOR of 5 business days, calculate the time series of 5
day P\&L's of this portfolio, starting on November 1, 2006.\\
\textit{Solution}: See attached spreadsheet \textit{Problem1 VaR Model Equity Portfolio.xlsx} analysis tab.\\

\textbf{(iii)} Run the EWMA model with $\lambda=0.97$ and $r_t=5$ day portfolio P\&L.
Remember that you need an extra 6 month (or so) period to prime the model. 
Scale the time series of the portfolio P\&L by $\sigma_t$ in order to produce a normalized series
(make sure that you use the correct units!).\\
\textit{Solution}: See attached spreadsheet \textit{Problem1 VaR Model Equity Portfolio.xlsx} analysis tab.\\

\textbf{(iv)} Assuming a rolling look-back period of 2 years (defined as 500 business days), 
calculate the portfolio VaR and CoVaR at the 99\% level of confidence for each of the days starting
 on November 1, 2006.\\
 \textit{Solution}: See attached spreadsheet \textit{Problem1 VaR Model Equity Portfolio.xlsx} analysis tab.\\
  
\textbf{(v)} Back test your VaR model for the period from November 1, 2006. Do
a detailed statistical analysis of the VaR breaches, and link them to
specific historic market events. Would using CoVaR instead of VaR
mitigate the breaches?
 \textit{Solution}: See attached spreadsheet \textit{Problem1 VaR Model Equity Portfolio.xlsx} analysis tab.\\

\newpage

\section{A Basic Commodity Model}
Consider the commodity model discussed in class
\begin{align*}
S(t) &= \exp\left( f(t) + X(t) \right),\\
dX(t) &= \lambda\left(\theta - X(t)\right)dt + \sigma dW(t),
\end{align*}
where $f(t)$ is a deterministic periodic function, and $\lambda$, $\theta$, and $\theta$ are
constant parameters. The process $S(t)$ is the spot price of the commodity (this could be a very
basic model for crude oil).\\

\textbf{(i)} Show that
$$
X(t) = X_0 e^{-\lambda t} + \theta (1-e^{-\lambda t}) + \sigma\int_0^t e^{-\lambda (t-s)}dW(s)
$$
is a solution to the SDE for $X(t)$ with the initial condition $X(0) = X_0$. Calculate the expected
value $\mathbb{E}[X(t)]$ and variance $\text{Var}[X(t)]$ of $X(t)$.\\
\textit{Solution}: Apply It\^o's lemma to $e^{\lambda t}X(t)$,
\begin{align*}
d\left( e^{\lambda t}X(t) \right) &= \lambda e^{\lambda t}  X(t) dt +  e^{\lambda t} dX(t) \\
&= e^{\lambda t} \left( \lambda X(t)dt + dX(t) \right)\\
&= e^{\lambda t} \left[ \lambda X(t)dt + \lambda\left(\theta - X(t)\right)dt + \sigma dW(t) \right]\\
&= e^{\lambda t} \left( \lambda\theta dt + \sigma dW(t) \right).
\end{align*}
Integrate from $0$ to $t$,
\begin{align*}
e^{\lambda t}X(t) - X(0) &=  \theta \int_0^t e^{\lambda s} d(\lambda s) + \sigma \int_0^t e^{\lambda s}dW(s)\\
& = \theta \left( e^{\lambda t} -1 \right) + \sigma \int_0^t e^{\lambda s}dW(s).
\end{align*}
Re-arrange the terms, we get
\begin{align}\label{xt}
X(t) = X_0 e^{-\lambda t} + \theta \left( 1- e^{-\lambda t} \right) + \sigma \int_0^t e^{-\lambda (t - s)}dW(s).
\end{align}
Given the martingality of the term $\sigma \int_0^t e^{-\lambda (t - s)}dW(s)\sim N\left(0, \sigma^2\int_0^t e^{-2\lambda (t - s)}ds\right)$, we get
\begin{align*}
\mathbb{E}\left[X(t)\right] &= X_0 e^{-\lambda t} + \theta \left( 1- e^{-\lambda t} \right) \xrightarrow[]{t \rightarrow \infty} \theta,\\
\text{Var}\left[X(t)\right] &= \frac{\sigma^2}{2\lambda}\left( 1 - e^{-2\lambda t} \right)\xrightarrow[]{t \rightarrow \infty}  \frac{\sigma^2}{2\lambda}.
\end{align*}\\

\textbf{(ii)} Consider the forward price on oil for the delivery on date $T$. Its time $t$ price, where $t<T$, is defined by
$$
F_T(t) = \mathbb{E}_t\left[S(T)\right].
$$
Find an explicit expression for $F_T(t)$ by calculating the expected value above.\\
\textit{Solution}: Conditioned on information available up to time $t$,
\begin{align*}
X(T) &= X(t)e^{-\lambda(T-t)}+ \theta \left( 1- e^{-\lambda (T-t)} \right)+ \sigma \int_t^T e^{-\lambda (T - s)}dW(s).\\
&\sim N\left( X(t)e^{-\lambda(T-t)}+ \theta \left( 1- e^{-\lambda (T-t)} \right), \frac{\sigma^2}{2\lambda}\left( 1 - e^{-2\lambda (T-t)} \right) \right),
\end{align*}
where $S(t) = \exp\left( f(t) + X(t) \right)$. Thus,
\begin{align*}
\mathbb{E}_t \left[\exp\left(  X(T) \right)\right] &= \exp\left(\mathbb{E}_t \left[ X(T)\right] + \frac{1}{2}\text{Var}\left[ X(T)\right]\right)\\
&= \exp\left( X(t)e^{-\lambda(T-t)}+ \theta \left( 1- e^{-\lambda (T-t)} \right) + \frac{\sigma^2}{4\lambda}\left( 1 - e^{-2\lambda (T-t)} \right)\right),
\end{align*}
and
\begin{align}
\nonumber F_T(t) &= \mathbb{E}_t\left[S(T)\right]\\
\nonumber &= \mathbb{E}_t\left[\exp\left(f(T) + X(T)\right)\right]\\
\label{ft}&= \exp\left( f(T) + X(t)e^{-\lambda(T-t)}+ \theta \left( 1- e^{-\lambda (T-t)} \right) + \frac{\sigma^2}{4\lambda}\left( 1 - e^{-2\lambda (T-t)} \right)\right)\\
\nonumber &= S(t)^{e^{-\lambda(T-t)}}\exp\left( f(T) - f(t)e^{-\lambda(T-t)}+ \theta \left( 1- e^{-\lambda (T-t)} \right) + \frac{\sigma^2}{4\lambda}\left( 1 - e^{-2\lambda (T-t)} \right)\right).
\end{align}\\

\textbf{(iii)} Consider the payer forward rate agreement (FRA) on crude oil on which the bank (B) agrees to buy crude at $T$ at the price $K$. Find
the risk neutral price $V_{\text{pay}}(t)$ of this contract.\\
\textit{Solution}: The pay off of the payer forward rate agreement at time $T$ is $S(T)-K$. The risk-neutral price of the FRA contract 
is equal to
\begin{align*}
V_{\text{pay}}(t) &= e^{-r(T-t)}\mathbb{E}_t\left[ S(T) - K \right]  = e^{-r(T-t)}\left(\mathbb{E}_t\left[ S(T) \right] - K\right),
\end{align*}
where $r$ is the risk-free zero rate over the time period $t\rightarrow T$ and $\mathbb{E}_t\left[S(T)\right]$
is obtained in the previous problem.\\

\underline{Note} that this problem should not be reasoned in the same way as an equity forward contract, where the payoff can be
replicated by a portfolio consisting of a long position of one share of the underlying asset $S(t)$ and a short position
of $Ke^{-r(T-t)}$ cash at time $t$. According to the law of one price, the risk-neutral price of the forward contract 
is equal to the value of the replicating portfolio at time $t$,
\begin{align*}
V_{\text{pay}}(t) &= S(t) - K e^{-r(T-t)}.
\end{align*}
This above result based on the law of one price is in agreement with the fact that the risk-neutral 
expectation $e^{-r(T-t)}\mathbb{E}_t\left[ S(T) \right] = S(t)$. Either approach assumes that there 
is no cost associated with holding the underlying asset, e.g.,
an equity stock. For commodities, however, there is a convenience yield (cost, premium, etc.) associated with holding
an underlying product and the equality $e^{-r(T-t)}\mathbb{E}_t\left[ S(T) \right] = S(t)$ is no longer true. By buying
a forward contract at time $t$, the buyer locks in the price of purchasing the commodity underlying asset at price
$F_T(t)$ at time $T$. When settled at time $t$, the payoff of
the FRA contract at time $T$ is $F_T(t)-K$, effectively equivalent to that of a fixed income instrument.\\



\textbf{(iv)} Recall the stochastic intensity model introduced in Problem \#3 of Assignment \#1:
$$
d\lambda_C (t) = \mu_C dt + \nu_C dZ_C(t),
$$
where $Z_C(t)$ is a Brownian motion. Assume that this model describes the credit of the bank's counterparty (C).\\
\textit{Solution}: OK. Recalled. \smiley{} The survival probability is
\begin{align}\label{survival}
 S(t,T) &= e^{- \lambda_{C}(0) (T-t) - \frac{1}{2}\mu_C  (T-t)^2 +\frac{1}{6}\nu_C^2 (T-t)^3 }.
\end{align}

\textbf{(v)} Assume that $W(t)$ and $Z_C(t)$ are uncorrelated (no wrong way risk). Assming that the bank is default-free, 
calculate explicitly the CVA on the oil FRA (use the discretized CVA formula) and interpret the expression in option
theoretic terms.\\
\textit{Solution}: According to the discretized CVA formula without wrong way risk, the
expected exposure and the counterparty default probability factorize:
\begin{align*}
\text{CVA}(t,T) &= (1-\bar{R})\sum_{i=1}^m \text{EE}_t(T_i,T)\left[ Q(t,T_i) - Q(t,T_{i-1}) \right]\\
&= (1-\bar{R})\sum_{i=1}^m \text{EE}_t(T_i,T)\left[  S(t,T_{i-1}) - S(t,T_i)\right],
\end{align*}
where $T_0 = t$ and $T_m = T$. The discounted expected exposure
\begin{align*}
\text{EE}_t(T_i,T) &\triangleq \mathbb{E}^{\mathbb{Q}}[P(t,T_i)V(T_i,T)^+]\\
 &= P(t,T_i)\mathbb{E}^{\mathbb{Q}}[V(T_i,T)^+],
\end{align*}
where we assumed that the discount factor $P(t,T_i) \perp V(T_i,T)$, and $V(T_i,T)$ is obtained in Part \textbf{(iii)}:
\begin{align*}
V(T_i,T) &= V_{\text{pay}}(T_i)\\
&=  P(T_i,T)\left(\mathbb{E}_{T_i}\left[ S(T) \right] - K\right)\\
&=  P(T_i,T)\left(F_T(T_i)- K\right),
\end{align*}
where $F_T(t)$ is obtained in Part \textbf{(ii)}. Thus, write $P(t,T_i)P(T_i,T)=P(t,T)$,
\begin{align*}
\text{EE}_t(T_i,T) &= P(t,T)\mathbb{E}^{\mathbb{Q}}[ \left(F_T(T_i)- K\right)^+],
\end{align*}
and
\begin{align*}
\text{CVA}(t,T) &=  (1-\bar{R})P(t,T)\sum_{i=1}^m  \mathbb{E}^{\mathbb{Q}}[ \left(F_T(T_i)- K\right)^+]\cdot\left[  S(t,T_{i-1}) - S(t,T_i)\right].
\end{align*}
Notice that $\mathbb{E}^{\mathbb{Q}}[ \left(F_T(T_i)- K\right)^+]$ is the risk-neutral expectation of the payoff of 
a call option of an FRA contract settled at $T_i$ to buy crude at $T$ at strike price $K$. Thus, the CVA on the crude oil FRA can be
interpreted as the present value of the unrecovered part of a sum of call options on FRA contracts, settled at intermediate times between $[t,T]$ 
and weighted by default probabilities for each time periods respectively. Effectively, the bank sells a series of call options to the
counterparty to be able to default, at various times between $[t,T]$.\\

\textbf{(vi)} Assume now $W(t)$ and $Z_C(t)$ are correlated (the counterparty may be an oil company), i.e.
$dW(t)dZ_C(t)=\rho dt$. Calculate the CVA under this assumption. What is the impact of the wrong way risk, depending on
the sign of $\rho$?\\
\textit{Solution}: Similar to the Part \textbf{(v)} but the exposure and the counterparty default probability no longer factorize:
\begin{align*}
\text{CVA}(t,T) &=  (1-\bar{R})P(t,T)\sum_{i=1}^m  
\mathbb{E}^{\mathbb{Q}}\left[ \left(F_T(T_i)- K\right)^+ \cdot   \left(\mathcal{S}(t,T_{i-1}) - \mathcal{S}(t,T_i)\right)| \tau=T_i\right],
\end{align*}
where $\mathcal{S}(t,T)$ denotes a random variable instead of expectation value as in Eq. \ref{survival}: $\mathcal{S}(t,T) \triangleq \exp\left( - \int_t^T\lambda_C(s)ds\right)$. Thus, according to Eq. \ref{ft},
\begin{align*}
\text{CVA}(t,T) &=  (1-\bar{R})P(t,T)\\
\times\sum_{i=1}^m 
\mathbb{E}^{\mathbb{Q}}\Bigg[ &\left(  
\exp\left( f(T) + X(T_i)e^{-\lambda(T-T_i)}+ \theta \left( 1- e^{-\lambda (T-T_i)} \right) + \frac{\sigma^2}{4\lambda}\left( 1 - e^{-2\lambda (T-T_i)} \right)\right)
-K \right)^+\\
\times & \left(\exp\left( - \int_t^{T_{i-1}}\lambda_C(s)ds\right) - \exp\left( - \int_t^{T_i}\lambda_C(s)ds\right)\right)\Bigg| \tau=T_i\Bigg],
\end{align*}
where according to Eq. \ref{xt},
\begin{align*}
X(T_i)e^{-\lambda(T-T_i)} &= X_0 e^{-\lambda T } + \theta  \left( e^{-\lambda (T-T_i)} - e^{-\lambda T} \right) + \sigma\int_0^{T_i}  e^{-\lambda (T-s)}dW(s)\\
\int_t^{T_i}\lambda_C(s)ds &= \lambda_0 (T_i-t) + \frac{1}{2}\theta (T_i-t)^2 + \sigma\int_t^{T_i} Z_C(s)ds.
\end{align*}
Written in a fully explicit way,
{
\footnotesize{
\begin{align*}
&\text{CVA}(t,T) =  (1-\bar{R})P(t,T)\\
 &\times\sum_{i=1}^m 
\mathbb{E}^{\mathbb{Q}}\left[\left(  
\exp\left( \underbrace{f(T) + X_0 e^{-\lambda T } +  \theta \left( 1- e^{-\lambda T}  \right) + \frac{\sigma^2}{4\lambda}\left( 1 - e^{-2\lambda (T-T_i)} \right) }_{\text{deterministic}}
+ \sigma\int_0^{T_i}  e^{-\lambda (T-s)}dW(s)\right)
-K \right)^+ \right.\\ 
& \left. \phantom{}\times  \left(\exp\left( \underbrace{- \lambda_0 (T_{i-1}-t) - \frac{1}{2}\theta (T_{i-1}-t)^2}_{\text{deterministic}} - \sigma\int_t^{T_{i-1}} Z_C(s)ds \right) \phantom{}- \exp\left( \underbrace{- \lambda_0 (T_i-t) - \frac{1}{2}\theta (T_i-t)^2}_{\text{deterministic}} - \sigma\int_t^{T_i} Z_C(s)ds  \right)\right)\Bigg| \tau=T_i\right],
\end{align*}
}
}
where $dW(t)dZ_C(t)=\rho dt$. The impact of the wrong way risk depending on the sign of $\rho$ is unclear from the above expression only.



\textbf{(vii)} Assume that the bank's credit is given by a stochastic intensity model
with the same specification but different parameters as in part \textbf{(iv)}:
$$
d\lambda_B (t) = \mu_B dt + \nu_B dZ_B(t).
$$
Assume that the pairs of Brownian motions $W(t)$ and $Z_B(t)$, and $Z_C(t)$ and $Z_B(t)$ are independent.
Derive an expression for BCVA on the oil payer FRA. Discuss the results (provide suitable graphs) for various
ranges of the model parameters.\\
\textit{Solution}: Assume there is no wrong way risk, the bilateral CVA formula
\begin{align*}
\text{BCVA}(t,T) &= (1-\bar{R}_C)\sum_{i=1}^m \text{EE}_t(T_i,T) S_B(t,T_{i-1})\left[ Q_C(t,T_i) - Q_C(t,T_{i-1}) \right]\\
&\quad +(1-\bar{R}_B)\sum_{i=1}^m \text{NEE}_t(T_i,T)S_C(t,T_{i-1}) \left[ Q_B(t,T_i) - Q_B(t,T_{i-1}) \right] \\
&= (1-\bar{R}_C)\sum_{i=1}^m \mathbb{E}^{\mathbb{Q}}[P(t,T_i)V(T_i, T)^+] S_B(t,T_{i-1})\left[ Q_C(t,T_i) - Q_C(t,T_{i-1}) \right]\\
&\quad +(1-\bar{R}_B)\sum_{i=1}^m \mathbb{E}^{\mathbb{Q}}[P(t,T_i)V(T_i, T)^-] S_C(t,T_{i-1})\left[ Q_B(t,T_i) - Q_B(t,T_{i-1}) \right] \\
&= (1-\bar{R}_C)P(t,T)\sum_{i=1}^m \mathbb{E}^{\mathbb{Q}}[(F_T(T_i)-K)^+] S_B(t,T_{i-1})\left[ Q_C(t,T_i) - Q_C(t,T_{i-1}) \right]\\
&\quad -(1-\bar{R}_B)P(t,T)\sum_{i=1}^m \mathbb{E}^{\mathbb{Q}}[(K-F_T(T_i))^+] S_C(t,T_{i-1})\left[ Q_B(t,T_i) - Q_B(t,T_{i-1}) \right],
\end{align*}
where $S_B(t,T)$, $S_C(t,T)$, $Q_B(t,T)$, and $Q_C(t,T)$ are defined accordingly.



\end{document}